\documentclass[]{ltjsarticle}
\usepackage{lmodern}
\usepackage{amssymb,amsmath}
\usepackage{ifxetex,ifluatex}
\usepackage{fixltx2e} % provides \textsubscript
\ifnum 0\ifxetex 1\fi\ifluatex 1\fi=0 % if pdftex
  \usepackage[T1]{fontenc}
  \usepackage[utf8]{inputenc}
\else % if luatex or xelatex
  \ifxetex
    \usepackage{mathspec}
  \else
    \usepackage{fontspec}
  \fi
  \defaultfontfeatures{Ligatures=TeX,Scale=MatchLowercase}
\fi
% use upquote if available, for straight quotes in verbatim environments
\IfFileExists{upquote.sty}{\usepackage{upquote}}{}
% use microtype if available
\IfFileExists{microtype.sty}{%
\usepackage{microtype}
\UseMicrotypeSet[protrusion]{basicmath} % disable protrusion for tt fonts
}{}
\usepackage[margin=1in]{geometry}
\usepackage{hyperref}
\hypersetup{unicode=true,
            pdftitle={rnnpsycholing Japanese NPI (no embedded sentences)},
            pdfauthor={Takashi Morita},
            pdfborder={0 0 0},
            breaklinks=true}
\urlstyle{same}  % don't use monospace font for urls
\usepackage{color}
\usepackage{fancyvrb}
\newcommand{\VerbBar}{|}
\newcommand{\VERB}{\Verb[commandchars=\\\{\}]}
\DefineVerbatimEnvironment{Highlighting}{Verbatim}{commandchars=\\\{\}}
% Add ',fontsize=\small' for more characters per line
\usepackage{framed}
\definecolor{shadecolor}{RGB}{248,248,248}
\newenvironment{Shaded}{\begin{snugshade}}{\end{snugshade}}
\newcommand{\KeywordTok}[1]{\textcolor[rgb]{0.13,0.29,0.53}{\textbf{#1}}}
\newcommand{\DataTypeTok}[1]{\textcolor[rgb]{0.13,0.29,0.53}{#1}}
\newcommand{\DecValTok}[1]{\textcolor[rgb]{0.00,0.00,0.81}{#1}}
\newcommand{\BaseNTok}[1]{\textcolor[rgb]{0.00,0.00,0.81}{#1}}
\newcommand{\FloatTok}[1]{\textcolor[rgb]{0.00,0.00,0.81}{#1}}
\newcommand{\ConstantTok}[1]{\textcolor[rgb]{0.00,0.00,0.00}{#1}}
\newcommand{\CharTok}[1]{\textcolor[rgb]{0.31,0.60,0.02}{#1}}
\newcommand{\SpecialCharTok}[1]{\textcolor[rgb]{0.00,0.00,0.00}{#1}}
\newcommand{\StringTok}[1]{\textcolor[rgb]{0.31,0.60,0.02}{#1}}
\newcommand{\VerbatimStringTok}[1]{\textcolor[rgb]{0.31,0.60,0.02}{#1}}
\newcommand{\SpecialStringTok}[1]{\textcolor[rgb]{0.31,0.60,0.02}{#1}}
\newcommand{\ImportTok}[1]{#1}
\newcommand{\CommentTok}[1]{\textcolor[rgb]{0.56,0.35,0.01}{\textit{#1}}}
\newcommand{\DocumentationTok}[1]{\textcolor[rgb]{0.56,0.35,0.01}{\textbf{\textit{#1}}}}
\newcommand{\AnnotationTok}[1]{\textcolor[rgb]{0.56,0.35,0.01}{\textbf{\textit{#1}}}}
\newcommand{\CommentVarTok}[1]{\textcolor[rgb]{0.56,0.35,0.01}{\textbf{\textit{#1}}}}
\newcommand{\OtherTok}[1]{\textcolor[rgb]{0.56,0.35,0.01}{#1}}
\newcommand{\FunctionTok}[1]{\textcolor[rgb]{0.00,0.00,0.00}{#1}}
\newcommand{\VariableTok}[1]{\textcolor[rgb]{0.00,0.00,0.00}{#1}}
\newcommand{\ControlFlowTok}[1]{\textcolor[rgb]{0.13,0.29,0.53}{\textbf{#1}}}
\newcommand{\OperatorTok}[1]{\textcolor[rgb]{0.81,0.36,0.00}{\textbf{#1}}}
\newcommand{\BuiltInTok}[1]{#1}
\newcommand{\ExtensionTok}[1]{#1}
\newcommand{\PreprocessorTok}[1]{\textcolor[rgb]{0.56,0.35,0.01}{\textit{#1}}}
\newcommand{\AttributeTok}[1]{\textcolor[rgb]{0.77,0.63,0.00}{#1}}
\newcommand{\RegionMarkerTok}[1]{#1}
\newcommand{\InformationTok}[1]{\textcolor[rgb]{0.56,0.35,0.01}{\textbf{\textit{#1}}}}
\newcommand{\WarningTok}[1]{\textcolor[rgb]{0.56,0.35,0.01}{\textbf{\textit{#1}}}}
\newcommand{\AlertTok}[1]{\textcolor[rgb]{0.94,0.16,0.16}{#1}}
\newcommand{\ErrorTok}[1]{\textcolor[rgb]{0.64,0.00,0.00}{\textbf{#1}}}
\newcommand{\NormalTok}[1]{#1}
\usepackage{graphicx,grffile}
\makeatletter
\def\maxwidth{\ifdim\Gin@nat@width>\linewidth\linewidth\else\Gin@nat@width\fi}
\def\maxheight{\ifdim\Gin@nat@height>\textheight\textheight\else\Gin@nat@height\fi}
\makeatother
% Scale images if necessary, so that they will not overflow the page
% margins by default, and it is still possible to overwrite the defaults
% using explicit options in \includegraphics[width, height, ...]{}
\setkeys{Gin}{width=\maxwidth,height=\maxheight,keepaspectratio}
\IfFileExists{parskip.sty}{%
\usepackage{parskip}
}{% else
\setlength{\parindent}{0pt}
\setlength{\parskip}{6pt plus 2pt minus 1pt}
}
\setlength{\emergencystretch}{3em}  % prevent overfull lines
\providecommand{\tightlist}{%
  \setlength{\itemsep}{0pt}\setlength{\parskip}{0pt}}
\setcounter{secnumdepth}{0}
% Redefines (sub)paragraphs to behave more like sections
\ifx\paragraph\undefined\else
\let\oldparagraph\paragraph
\renewcommand{\paragraph}[1]{\oldparagraph{#1}\mbox{}}
\fi
\ifx\subparagraph\undefined\else
\let\oldsubparagraph\subparagraph
\renewcommand{\subparagraph}[1]{\oldsubparagraph{#1}\mbox{}}
\fi

%%% Use protect on footnotes to avoid problems with footnotes in titles
\let\rmarkdownfootnote\footnote%
\def\footnote{\protect\rmarkdownfootnote}

%%% Change title format to be more compact
\usepackage{titling}

% Create subtitle command for use in maketitle
\newcommand{\subtitle}[1]{
  \posttitle{
    \begin{center}\large#1\end{center}
    }
}

\setlength{\droptitle}{-2em}
  \title{rnnpsycholing Japanese NPI (no embedded sentences)}
  \pretitle{\vspace{\droptitle}\centering\huge}
  \posttitle{\par}
  \author{Takashi Morita}
  \preauthor{\centering\large\emph}
  \postauthor{\par}
  \date{}
  \predate{}\postdate{}


\begin{document}
\maketitle

{
\setcounter{tocdepth}{3}
\tableofcontents
}
\section{Introduction}\label{introduction}

We are looking for a 2x2 interaction of:

\begin{itemize}
\tightlist
\item
  presence vs.~absence of the Japanese NPI \emph{shika} (しか)
\item
  affirmativeness vs.~negativeness of main verb
\end{itemize}

for each of the three grammatical cases (TOP, ACC, DAT) of the
\emph{shika}-attached NP.

e.g.

\begin{itemize}
\tightlist
\item
  TOP

  \begin{itemize}
  \tightlist
  \item
    渡辺-\{しか,は\} 家族-に 手紙-を \{渡した,渡さなかった\}。
  \item
    Watanabe-\{\emph{shika},TOP\} family-DAT letter-ACC \{came, didn't
    come\}.
  \item
    `Only Watanabe handed letters to his family.'
  \item
    `Watanabe handed/didn't hand letters to his family.'
  \end{itemize}
\item
  ACC

  \begin{itemize}
  \tightlist
  \item
    渡辺-は 家族-に 手紙-\{しか,を\} \{渡した,渡さなかった\}。
  \item
    Watanabe-TOP family-DAT letter-\{\emph{shika},ACC\} \{came,
    didn't.come.\}
  \item
    `Watanabe handed only letters to his family.'
  \item
    `Watanabe handed/didn't hand letters to his family.'
  \end{itemize}
\item
  DAT

  \begin{itemize}
  \tightlist
  \item
    渡辺-は 家族-\{に-しか,に\} 手紙-を \{渡した,渡さなかった\}。
  \item
    Watanabe-TOP family-\{DAT-\emph{shika},DAT\} letter-ACC \{came,
    didn't.come.\}
  \item
    `Watanabe handed letters only to his family.'
  \item
    `Watanabe handed/didn't hand letters to his family.'
  \end{itemize}
\end{itemize}

Why is this interesting?

\begin{itemize}
\tightlist
\item
  A grammatical sentence with \emph{shika} must have a negative verb.
\item
  Affirmative verbs would show significant increase in surprisal when
  \emph{shika} precedes compared with its absence.
\end{itemize}

\section{Methods}\label{methods}

For each pair \(i\) of sentences with vs.~without \emph{shika}, we look
at their difference in surprisal of the verb (\texttt{V}) region.

\[
D_i
     := 
        S(\texttt{V}_i \mid \texttt{shika})
        -
        S(\texttt{V}_i \mid \texttt{no-shika})
\]

\[
S(r) := - \log_2 P(r)
\]

And we perform a statistical analysis and check if the affirmativeness
vs.~negativeness of the verb have an effect on the surprisal difference
\(D\).

\section{Load data}\label{load-data}

\begin{Shaded}
\begin{Highlighting}[]
\KeywordTok{rm}\NormalTok{(}\DataTypeTok{list =} \KeywordTok{ls}\NormalTok{())}
\KeywordTok{library}\NormalTok{(tidyverse)}
\KeywordTok{library}\NormalTok{(brms)}
\KeywordTok{library}\NormalTok{(lme4)}
\KeywordTok{library}\NormalTok{(lmerTest)}
\KeywordTok{library}\NormalTok{(plotrix)}


\NormalTok{REGIONS =}\StringTok{ }\KeywordTok{c}\NormalTok{(}\StringTok{'prefix'}\NormalTok{, }\StringTok{'V'}\NormalTok{, }\StringTok{'end'}\NormalTok{)}


\NormalTok{token_based_data_path =}\StringTok{ 'jp_shika_test_sentences_unembedded_surprisal-per-token.tsv'}
\NormalTok{data_token_based =}\StringTok{ }\KeywordTok{read_tsv}\NormalTok{(token_based_data_path)}
\end{Highlighting}
\end{Shaded}

\begin{verbatim}
## Parsed with column specification:
## cols(
##   sent_index = col_integer(),
##   token_index = col_integer(),
##   token = col_character(),
##   region = col_character(),
##   log_prob = col_double(),
##   shika_case = col_character(),
##   shika_embedded = col_character(),
##   other_v_type = col_character(),
##   shika = col_character(),
##   verb_type = col_character(),
##   surprisal = col_double(),
##   LSTM = col_character()
## )
\end{verbatim}

\begin{Shaded}
\begin{Highlighting}[]
\CommentTok{# Fill the initial surprisal by 0.}
\NormalTok{data_token_based[}\KeywordTok{is.na}\NormalTok{(data_token_based}\OperatorTok{$}\NormalTok{surprisal),]}\OperatorTok{$}\NormalTok{surprisal =}\StringTok{ }\DecValTok{0}
\NormalTok{data_token_based}\OperatorTok{$}\NormalTok{region =}\StringTok{ }\KeywordTok{factor}\NormalTok{(data_token_based}\OperatorTok{$}\NormalTok{region, }\DataTypeTok{levels=}\NormalTok{REGIONS)}



\NormalTok{data_region_based =}\StringTok{ }\NormalTok{data_token_based }\OperatorTok\StringTok{ }
\StringTok{    }\KeywordTok{group_by}\NormalTok{(sent_index, region, shika, verb_type, shika_case) }\OperatorTok\StringTok{ }
\StringTok{        }\KeywordTok{summarise}\NormalTok{(}\DataTypeTok{surprisal=}\KeywordTok{sum}\NormalTok{(surprisal)) }\OperatorTok
\StringTok{        }\KeywordTok{ungroup}\NormalTok{() }\OperatorTok\StringTok{ }
\StringTok{    }\KeywordTok{mutate}\NormalTok{(}
        \DataTypeTok{shika=}\KeywordTok{factor}\NormalTok{(shika, }\DataTypeTok{levels=}\KeywordTok{c}\NormalTok{(}\StringTok{"shika"}\NormalTok{, }\StringTok{"no-shika"}\NormalTok{)),}
        \DataTypeTok{verb_type=}\KeywordTok{factor}\NormalTok{(verb_type, }\DataTypeTok{levels=}\KeywordTok{c}\NormalTok{(}\StringTok{"affirmative"}\NormalTok{, }\StringTok{"negative"}\NormalTok{)),}
        \DataTypeTok{shika_case=}\KeywordTok{factor}\NormalTok{(shika_case, }\DataTypeTok{levels=}\KeywordTok{c}\NormalTok{(}\StringTok{"TOP"}\NormalTok{, }\StringTok{"ACC"}\NormalTok{, }\StringTok{"DAT"}\NormalTok{))}
\NormalTok{        )}

\CommentTok{# Sum coding of the variables.}
\KeywordTok{contrasts}\NormalTok{(data_region_based}\OperatorTok{$}\NormalTok{shika) =}\StringTok{ "contr.sum"}
\KeywordTok{contrasts}\NormalTok{(data_region_based}\OperatorTok{$}\NormalTok{verb_type) =}\StringTok{ "contr.sum"}

\CommentTok{# Make sure that the dataframe is sorted appropriately.}
\CommentTok{# First by case}
\NormalTok{data_region_based =}\StringTok{ }\NormalTok{data_region_based[}\KeywordTok{order}\NormalTok{(data_region_based}\OperatorTok{$}\NormalTok{shika_case),]}
\CommentTok{# Second by verb_type (affirmative vs. negative)}
\NormalTok{data_region_based =}\StringTok{ }\NormalTok{data_region_based[}\KeywordTok{order}\NormalTok{(data_region_based}\OperatorTok{$}\NormalTok{verb_type),]}
\CommentTok{# Then by sent_index}
\NormalTok{data_region_based =}\StringTok{ }\NormalTok{data_region_based[}\KeywordTok{order}\NormalTok{(data_region_based}\OperatorTok{$}\NormalTok{sent_index),]}
\end{Highlighting}
\end{Shaded}

\section{Visualization}\label{visualization}

\begin{Shaded}
\begin{Highlighting}[]
\CommentTok{# Focus on the V (verb) region.}
\NormalTok{data_V =}\StringTok{ }\KeywordTok{subset}\NormalTok{(data_region_based, region }\OperatorTok{==}\StringTok{ 'V'}\NormalTok{)}

\CommentTok{# Get difference in surprisal between shika vs. no-shika.}
\NormalTok{data_V_shika =}\StringTok{ }\KeywordTok{subset}\NormalTok{(data_V, shika }\OperatorTok{==}\StringTok{ 'shika'}\NormalTok{)}
\NormalTok{data_V_no_shika =}\StringTok{ }\KeywordTok{subset}\NormalTok{(data_V, shika }\OperatorTok{==}\StringTok{ 'no-shika'}\NormalTok{)}
\NormalTok{data_V_shika}\OperatorTok{$}\NormalTok{surprisal_diff =}\StringTok{ }\NormalTok{data_V_shika}\OperatorTok{$}\NormalTok{surprisal }\OperatorTok{-}\StringTok{ }\NormalTok{data_V_no_shika}\OperatorTok{$}\NormalTok{surprisal}

\CommentTok{# Visualize the difference in surprisal increase/dicrease between affirmative vs. negative verbs.}
\NormalTok{data_V_shika }\OperatorTok\StringTok{ }
\StringTok{    }\KeywordTok{group_by}\NormalTok{(verb_type, shika_case) }\OperatorTok
\StringTok{    }\KeywordTok{summarise}\NormalTok{(}\DataTypeTok{m=}\KeywordTok{mean}\NormalTok{(surprisal_diff),}
            \DataTypeTok{s=}\KeywordTok{std.error}\NormalTok{(surprisal_diff),}
            \DataTypeTok{upper=}\NormalTok{m }\OperatorTok{+}\StringTok{ }\FloatTok{1.96}\OperatorTok{*}\NormalTok{s,}
            \DataTypeTok{lower=}\NormalTok{m }\OperatorTok{-}\StringTok{ }\FloatTok{1.96}\OperatorTok{*}\NormalTok{s) }\OperatorTok
\StringTok{    }\KeywordTok{ungroup}\NormalTok{() }\OperatorTok
\StringTok{    }\KeywordTok{ggplot}\NormalTok{(}\KeywordTok{aes}\NormalTok{(}\DataTypeTok{x=}\NormalTok{shika_case, }\DataTypeTok{y=}\NormalTok{m, }\DataTypeTok{ymin=}\NormalTok{lower, }\DataTypeTok{ymax=}\NormalTok{upper, }\DataTypeTok{width=}\FloatTok{0.4}\NormalTok{, }\DataTypeTok{fill=}\NormalTok{verb_type)) }\OperatorTok{+}
\StringTok{        }\KeywordTok{geom_bar}\NormalTok{(}\DataTypeTok{stat =} \StringTok{'identity'}\NormalTok{, }\DataTypeTok{position =} \StringTok{"dodge"}\NormalTok{) }\OperatorTok{+}
\StringTok{        }\KeywordTok{geom_errorbar}\NormalTok{(}\DataTypeTok{position=}\KeywordTok{position_dodge}\NormalTok{(}\FloatTok{0.4}\NormalTok{), }\DataTypeTok{width=}\NormalTok{.}\DecValTok{1}\NormalTok{)}
\end{Highlighting}
\end{Shaded}

\includegraphics{analysis_unembedded_files/figure-latex/unnamed-chunk-2-1.pdf}
- TOP - No visible increase in surprisal of the affirmative verbs. -
Visible decrease in surprisal of the negative verbs. - ACC - Greatest
increase in surprisal of the affirmative verbs. - Small decrease in
surprisal of the negative verbs. - DAT - Visible increase in surprisal
of the affirmative verbs. - Greatest decrease in surprisal of the
negative verbs.

\section{Regressions}\label{regressions}

\subsection{TOP}\label{top}

\begin{Shaded}
\begin{Highlighting}[]
\NormalTok{sub_data =}\StringTok{ }\KeywordTok{subset}\NormalTok{(data_V_shika, shika_case }\OperatorTok{==}\StringTok{ 'TOP'}\NormalTok{)}

\NormalTok{m =}\StringTok{ }\KeywordTok{lmer}\NormalTok{(}
\NormalTok{        surprisal_diff}
            \OperatorTok{~}\StringTok{ }\NormalTok{verb_type}
                \OperatorTok{+}\StringTok{ }\NormalTok{(}\DecValTok{1} \OperatorTok{|}\StringTok{ }\NormalTok{sent_index)}
\NormalTok{        ,}
        \DataTypeTok{data=}\NormalTok{sub_data}
\NormalTok{        )}
\KeywordTok{summary}\NormalTok{(m)}
\end{Highlighting}
\end{Shaded}

\begin{verbatim}
## Linear mixed model fit by REML. t-tests use Satterthwaite's method [
## lmerModLmerTest]
## Formula: surprisal_diff ~ verb_type + (1 | sent_index)
##    Data: sub_data
## 
## REML criterion at convergence: 376.7
## 
## Scaled residuals: 
##      Min       1Q   Median       3Q      Max 
## -2.39656 -0.38524 -0.04051  0.46275  2.27091 
## 
## Random effects:
##  Groups     Name        Variance Std.Dev.
##  sent_index (Intercept) 1.393    1.180   
##  Residual               1.843    1.358   
## Number of obs: 96, groups:  sent_index, 48
## 
## Fixed effects:
##             Estimate Std. Error      df t value Pr(>|t|)    
## (Intercept)  -0.6617     0.2196 47.0000  -3.013 0.004159 ** 
## verb_type1    0.5436     0.1386 47.0000   3.923 0.000283 ***
## ---
## Signif. codes:  0 '***' 0.001 '**' 0.01 '*' 0.05 '.' 0.1 ' ' 1
## 
## Correlation of Fixed Effects:
##            (Intr)
## verb_type1 0.000
\end{verbatim}

\begin{itemize}
\tightlist
\item
  Statistically significant effect of the affirmativeness
  vs.~negativeness of verbs.
\end{itemize}

\subsection{ACC}\label{acc}

\begin{Shaded}
\begin{Highlighting}[]
\NormalTok{sub_data =}\StringTok{ }\KeywordTok{subset}\NormalTok{(data_V_shika, shika_case }\OperatorTok{==}\StringTok{ 'ACC'}\NormalTok{)}

\NormalTok{m =}\StringTok{ }\KeywordTok{lmer}\NormalTok{(}
\NormalTok{        surprisal_diff}
            \OperatorTok{~}\StringTok{ }\NormalTok{verb_type}
                \OperatorTok{+}\StringTok{ }\NormalTok{(}\DecValTok{1} \OperatorTok{|}\StringTok{ }\NormalTok{sent_index)}
\NormalTok{        ,}
        \DataTypeTok{data=}\NormalTok{sub_data}
\NormalTok{        )}
\KeywordTok{summary}\NormalTok{(m)}
\end{Highlighting}
\end{Shaded}

\begin{verbatim}
## Linear mixed model fit by REML. t-tests use Satterthwaite's method [
## lmerModLmerTest]
## Formula: surprisal_diff ~ verb_type + (1 | sent_index)
##    Data: sub_data
## 
## REML criterion at convergence: 171.4
## 
## Scaled residuals: 
##      Min       1Q   Median       3Q      Max 
## -2.03979 -0.47071  0.05871  0.53181  1.37674 
## 
## Random effects:
##  Groups     Name        Variance Std.Dev.
##  sent_index (Intercept) 6.79     2.606   
##  Residual               2.01     1.418   
## Number of obs: 38, groups:  sent_index, 19
## 
## Fixed effects:
##             Estimate Std. Error      df t value Pr(>|t|)    
## (Intercept)   2.0372     0.6405 18.0000   3.181  0.00518 ** 
## verb_type1    2.6455     0.2300 18.0000  11.503 9.94e-10 ***
## ---
## Signif. codes:  0 '***' 0.001 '**' 0.01 '*' 0.05 '.' 0.1 ' ' 1
## 
## Correlation of Fixed Effects:
##            (Intr)
## verb_type1 0.000
\end{verbatim}

\begin{itemize}
\tightlist
\item
  Statistically significant effect of the affirmativeness
  vs.~negativeness of verbs.
\item
  Greater effect than TOP.
\end{itemize}

\subsection{DAT}\label{dat}

\begin{Shaded}
\begin{Highlighting}[]
\NormalTok{sub_data =}\StringTok{ }\KeywordTok{subset}\NormalTok{(data_V_shika, shika_case }\OperatorTok{==}\StringTok{ 'DAT'}\NormalTok{)}

\NormalTok{m =}\StringTok{ }\KeywordTok{lmer}\NormalTok{(}
\NormalTok{        surprisal_diff}
            \OperatorTok{~}\StringTok{ }\NormalTok{verb_type}
                \OperatorTok{+}\StringTok{ }\NormalTok{(}\DecValTok{1} \OperatorTok{|}\StringTok{ }\NormalTok{sent_index)}
\NormalTok{        ,}
        \DataTypeTok{data=}\NormalTok{sub_data}
\NormalTok{        )}
\KeywordTok{summary}\NormalTok{(m)}
\end{Highlighting}
\end{Shaded}

\begin{verbatim}
## Linear mixed model fit by REML. t-tests use Satterthwaite's method [
## lmerModLmerTest]
## Formula: surprisal_diff ~ verb_type + (1 | sent_index)
##    Data: sub_data
## 
## REML criterion at convergence: 128.9
## 
## Scaled residuals: 
##     Min      1Q  Median      3Q     Max 
## -1.1309 -0.7259  0.1149  0.5097  1.3964 
## 
## Random effects:
##  Groups     Name        Variance Std.Dev.
##  sent_index (Intercept) 4.020    2.005   
##  Residual               1.256    1.121   
## Number of obs: 32, groups:  sent_index, 16
## 
## Fixed effects:
##             Estimate Std. Error       df t value Pr(>|t|)    
## (Intercept)  0.08783    0.53897 15.00000   0.163    0.873    
## verb_type1   2.67121    0.19811 15.00000  13.483 8.65e-10 ***
## ---
## Signif. codes:  0 '***' 0.001 '**' 0.01 '*' 0.05 '.' 0.1 ' ' 1
## 
## Correlation of Fixed Effects:
##            (Intr)
## verb_type1 0.000
\end{verbatim}

\begin{itemize}
\tightlist
\item
  Statistically significant effect of the affirmativeness
  vs.~negativeness of verbs.
\item
  Greater effect than TOP, similar to ACC.
\end{itemize}


\end{document}
