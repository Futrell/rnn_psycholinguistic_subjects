\documentclass[]{ltjsarticle}
\usepackage{lmodern}
\usepackage{amssymb,amsmath}
\usepackage{ifxetex,ifluatex}
\usepackage{fixltx2e} % provides \textsubscript
\ifnum 0\ifxetex 1\fi\ifluatex 1\fi=0 % if pdftex
  \usepackage[T1]{fontenc}
  \usepackage[utf8]{inputenc}
\else % if luatex or xelatex
  \ifxetex
    \usepackage{mathspec}
  \else
    \usepackage{fontspec}
  \fi
  \defaultfontfeatures{Ligatures=TeX,Scale=MatchLowercase}
\fi
% use upquote if available, for straight quotes in verbatim environments
\IfFileExists{upquote.sty}{\usepackage{upquote}}{}
% use microtype if available
\IfFileExists{microtype.sty}{%
\usepackage{microtype}
\UseMicrotypeSet[protrusion]{basicmath} % disable protrusion for tt fonts
}{}
\usepackage[margin=1in]{geometry}
\usepackage{hyperref}
\hypersetup{unicode=true,
            pdftitle={rnnpsycholing Japanese NPI (sentences with embedding and matrix shika)},
            pdfauthor={Takashi Morita},
            pdfborder={0 0 0},
            breaklinks=true}
\urlstyle{same}  % don't use monospace font for urls
\usepackage{color}
\usepackage{fancyvrb}
\newcommand{\VerbBar}{|}
\newcommand{\VERB}{\Verb[commandchars=\\\{\}]}
\DefineVerbatimEnvironment{Highlighting}{Verbatim}{commandchars=\\\{\}}
% Add ',fontsize=\small' for more characters per line
\usepackage{framed}
\definecolor{shadecolor}{RGB}{248,248,248}
\newenvironment{Shaded}{\begin{snugshade}}{\end{snugshade}}
\newcommand{\KeywordTok}[1]{\textcolor[rgb]{0.13,0.29,0.53}{\textbf{#1}}}
\newcommand{\DataTypeTok}[1]{\textcolor[rgb]{0.13,0.29,0.53}{#1}}
\newcommand{\DecValTok}[1]{\textcolor[rgb]{0.00,0.00,0.81}{#1}}
\newcommand{\BaseNTok}[1]{\textcolor[rgb]{0.00,0.00,0.81}{#1}}
\newcommand{\FloatTok}[1]{\textcolor[rgb]{0.00,0.00,0.81}{#1}}
\newcommand{\ConstantTok}[1]{\textcolor[rgb]{0.00,0.00,0.00}{#1}}
\newcommand{\CharTok}[1]{\textcolor[rgb]{0.31,0.60,0.02}{#1}}
\newcommand{\SpecialCharTok}[1]{\textcolor[rgb]{0.00,0.00,0.00}{#1}}
\newcommand{\StringTok}[1]{\textcolor[rgb]{0.31,0.60,0.02}{#1}}
\newcommand{\VerbatimStringTok}[1]{\textcolor[rgb]{0.31,0.60,0.02}{#1}}
\newcommand{\SpecialStringTok}[1]{\textcolor[rgb]{0.31,0.60,0.02}{#1}}
\newcommand{\ImportTok}[1]{#1}
\newcommand{\CommentTok}[1]{\textcolor[rgb]{0.56,0.35,0.01}{\textit{#1}}}
\newcommand{\DocumentationTok}[1]{\textcolor[rgb]{0.56,0.35,0.01}{\textbf{\textit{#1}}}}
\newcommand{\AnnotationTok}[1]{\textcolor[rgb]{0.56,0.35,0.01}{\textbf{\textit{#1}}}}
\newcommand{\CommentVarTok}[1]{\textcolor[rgb]{0.56,0.35,0.01}{\textbf{\textit{#1}}}}
\newcommand{\OtherTok}[1]{\textcolor[rgb]{0.56,0.35,0.01}{#1}}
\newcommand{\FunctionTok}[1]{\textcolor[rgb]{0.00,0.00,0.00}{#1}}
\newcommand{\VariableTok}[1]{\textcolor[rgb]{0.00,0.00,0.00}{#1}}
\newcommand{\ControlFlowTok}[1]{\textcolor[rgb]{0.13,0.29,0.53}{\textbf{#1}}}
\newcommand{\OperatorTok}[1]{\textcolor[rgb]{0.81,0.36,0.00}{\textbf{#1}}}
\newcommand{\BuiltInTok}[1]{#1}
\newcommand{\ExtensionTok}[1]{#1}
\newcommand{\PreprocessorTok}[1]{\textcolor[rgb]{0.56,0.35,0.01}{\textit{#1}}}
\newcommand{\AttributeTok}[1]{\textcolor[rgb]{0.77,0.63,0.00}{#1}}
\newcommand{\RegionMarkerTok}[1]{#1}
\newcommand{\InformationTok}[1]{\textcolor[rgb]{0.56,0.35,0.01}{\textbf{\textit{#1}}}}
\newcommand{\WarningTok}[1]{\textcolor[rgb]{0.56,0.35,0.01}{\textbf{\textit{#1}}}}
\newcommand{\AlertTok}[1]{\textcolor[rgb]{0.94,0.16,0.16}{#1}}
\newcommand{\ErrorTok}[1]{\textcolor[rgb]{0.64,0.00,0.00}{\textbf{#1}}}
\newcommand{\NormalTok}[1]{#1}
\usepackage{graphicx,grffile}
\makeatletter
\def\maxwidth{\ifdim\Gin@nat@width>\linewidth\linewidth\else\Gin@nat@width\fi}
\def\maxheight{\ifdim\Gin@nat@height>\textheight\textheight\else\Gin@nat@height\fi}
\makeatother
% Scale images if necessary, so that they will not overflow the page
% margins by default, and it is still possible to overwrite the defaults
% using explicit options in \includegraphics[width, height, ...]{}
\setkeys{Gin}{width=\maxwidth,height=\maxheight,keepaspectratio}
\IfFileExists{parskip.sty}{%
\usepackage{parskip}
}{% else
\setlength{\parindent}{0pt}
\setlength{\parskip}{6pt plus 2pt minus 1pt}
}
\setlength{\emergencystretch}{3em}  % prevent overfull lines
\providecommand{\tightlist}{%
  \setlength{\itemsep}{0pt}\setlength{\parskip}{0pt}}
\setcounter{secnumdepth}{0}
% Redefines (sub)paragraphs to behave more like sections
\ifx\paragraph\undefined\else
\let\oldparagraph\paragraph
\renewcommand{\paragraph}[1]{\oldparagraph{#1}\mbox{}}
\fi
\ifx\subparagraph\undefined\else
\let\oldsubparagraph\subparagraph
\renewcommand{\subparagraph}[1]{\oldsubparagraph{#1}\mbox{}}
\fi

%%% Use protect on footnotes to avoid problems with footnotes in titles
\let\rmarkdownfootnote\footnote%
\def\footnote{\protect\rmarkdownfootnote}

%%% Change title format to be more compact
\usepackage{titling}

% Create subtitle command for use in maketitle
\newcommand{\subtitle}[1]{
  \posttitle{
    \begin{center}\large#1\end{center}
    }
}

\setlength{\droptitle}{-2em}
  \title{rnnpsycholing Japanese NPI (sentences with embedding and matrix shika)}
  \pretitle{\vspace{\droptitle}\centering\huge}
  \posttitle{\par}
  \author{Takashi Morita}
  \preauthor{\centering\large\emph}
  \postauthor{\par}
  \date{}
  \predate{}\postdate{}


\begin{document}
\maketitle

{
\setcounter{tocdepth}{3}
\tableofcontents
}
\section{Introduction}\label{introduction}

We are looking for a 2x2x2 interaction of:

\begin{itemize}
\tightlist
\item
  presence vs.~absence of the Japanese NPI \emph{shika} (しか) in the
  main clause.
\item
  affirmativeness vs.~negativeness of main verb
\item
  affirmativeness vs.~negativeness of embedded verb
\end{itemize}

for each of the three grammatical cases (TOP, DAT) of the
\emph{shika}-attached NP.

e.g.

\begin{itemize}
\tightlist
\item
  TOP

  \begin{itemize}
  \tightlist
  \item
    佐藤-\{しか, は\} 社長-が パーティ-に 友人-を \{呼んだ,
    呼ばなかった\} と \{思った,思わなかった\}。
  \item
    Sato-\{\emph{shika}, TOP\} CEO-NOM party-DAT friend-ACC \{invited,
    didn't invite\} that \{thought, didn't think\}.
  \end{itemize}
\item
  DAT

  \begin{itemize}
  \tightlist
  \item
    同僚-\{にしか, に\} 佐藤は 社長-が パーティ-に 友人-を \{呼んだ,
    呼ばなかった\} と \{思った,思わなかった\}。
  \item
    colleague-\{DAT-\emph{shika}, DAT\} Sato-TOP CEO-NOM party-DAT
    friend-ACC \{invited, didn't invite\} that \{thought, didn't
    think\}.
  \end{itemize}
\end{itemize}

Why is this interesting?

\begin{enumerate}
\def\labelenumi{\arabic{enumi}.}
\tightlist
\item
  A grammatical sentence with \emph{shika} in the main clause must have
  a negative main verb.

  \begin{itemize}
  \tightlist
  \item
    A significant increase in surprisal of the affirmative main verbs
    must be predicted by the LSTM conditioned on the presence of
    \emph{shika} if the learning is successful.
  \end{itemize}
\item
  Negation of the embedded verb does not satisfy the \emph{shika}'s
  grammatical condition.

  \begin{itemize}
  \tightlist
  \item
    No significant increase in surprisal of the affirmative embedded
    verbs given \emph{shika} is expected for a successful learner.
  \item
    Nor significant interaction between the main and embedded verbs
    given \emph{shika} is expected for a successful learner.
  \end{itemize}
\end{enumerate}

\section{Load data}\label{load-data}

\begin{Shaded}
\begin{Highlighting}[]
\KeywordTok{rm}\NormalTok{(}\DataTypeTok{list =} \KeywordTok{ls}\NormalTok{())}
\KeywordTok{library}\NormalTok{(tidyverse)}
\KeywordTok{library}\NormalTok{(brms)}
\KeywordTok{library}\NormalTok{(lme4)}
\KeywordTok{library}\NormalTok{(lmerTest)}
\KeywordTok{library}\NormalTok{(plotrix)}

\NormalTok{REGIONS =}\StringTok{ }\KeywordTok{c}\NormalTok{(}\StringTok{'main_prefix'}\NormalTok{, }\StringTok{'embedded_prefix'}\NormalTok{, }\StringTok{'embedded_V'}\NormalTok{, }\StringTok{'complementizer'}\NormalTok{, }\StringTok{'main_V'}\NormalTok{, }\StringTok{'end'}\NormalTok{)}

\NormalTok{token_based_data_path =}\StringTok{ 'jp_shika_test_sentences_embedded_shika-in-main_surprisal-per-token.tsv'}
\NormalTok{data_token_based =}\StringTok{ }\KeywordTok{read_tsv}\NormalTok{(token_based_data_path)}
\end{Highlighting}
\end{Shaded}

\begin{verbatim}
## Parsed with column specification:
## cols(
##   sent_index = col_integer(),
##   token_index = col_integer(),
##   token = col_character(),
##   region = col_character(),
##   log_prob = col_double(),
##   shika_case = col_character(),
##   shika = col_character(),
##   embed_V = col_character(),
##   main_V = col_character(),
##   surprisal = col_double(),
##   LSTM = col_character()
## )
\end{verbatim}

\begin{Shaded}
\begin{Highlighting}[]
\CommentTok{# Fill the initial surprisal by 0.}
\NormalTok{data_token_based[}\KeywordTok{is.na}\NormalTok{(data_token_based}\OperatorTok{$}\NormalTok{surprisal),]}\OperatorTok{$}\NormalTok{surprisal =}\StringTok{ }\DecValTok{0}
\NormalTok{data_token_based}\OperatorTok{$}\NormalTok{region =}\StringTok{ }\KeywordTok{factor}\NormalTok{(data_token_based}\OperatorTok{$}\NormalTok{region, }\DataTypeTok{levels=}\NormalTok{REGIONS)}

\NormalTok{data_region_based =}\StringTok{ }\NormalTok{data_token_based }\OperatorTok\StringTok{ }
\StringTok{    }\KeywordTok{group_by}\NormalTok{(sent_index, region, shika, embed_V, main_V, shika_case) }\OperatorTok\StringTok{ }
\StringTok{        }\KeywordTok{summarise}\NormalTok{(}\DataTypeTok{surprisal=}\KeywordTok{sum}\NormalTok{(surprisal)) }\OperatorTok
\StringTok{        }\KeywordTok{ungroup}\NormalTok{() }\OperatorTok\StringTok{ }
\StringTok{    }\KeywordTok{mutate}\NormalTok{(}
        \DataTypeTok{shika=}\KeywordTok{factor}\NormalTok{(shika, }\DataTypeTok{levels=}\KeywordTok{c}\NormalTok{(}\StringTok{"shika"}\NormalTok{, }\StringTok{"no-shika"}\NormalTok{)),}
        \DataTypeTok{embed_V=}\KeywordTok{factor}\NormalTok{(embed_V, }\DataTypeTok{levels=}\KeywordTok{c}\NormalTok{(}\StringTok{"affirmative"}\NormalTok{, }\StringTok{"negative"}\NormalTok{)),}
        \DataTypeTok{main_V=}\KeywordTok{factor}\NormalTok{(main_V, }\DataTypeTok{levels=}\KeywordTok{c}\NormalTok{(}\StringTok{"affirmative"}\NormalTok{, }\StringTok{"negative"}\NormalTok{)),}
        \DataTypeTok{shika_case=}\KeywordTok{factor}\NormalTok{(shika_case, }\DataTypeTok{levels=}\KeywordTok{c}\NormalTok{(}\StringTok{"TOP"}\NormalTok{, }\StringTok{"DAT"}\NormalTok{))}
\NormalTok{        )}

\CommentTok{# Sum coding of the variables.}
\KeywordTok{contrasts}\NormalTok{(data_region_based}\OperatorTok{$}\NormalTok{shika) =}\StringTok{ "contr.sum"}
\KeywordTok{contrasts}\NormalTok{(data_region_based}\OperatorTok{$}\NormalTok{embed_V) =}\StringTok{ "contr.sum"}
\KeywordTok{contrasts}\NormalTok{(data_region_based}\OperatorTok{$}\NormalTok{main_V) =}\StringTok{ "contr.sum"}

\CommentTok{# Make sure that the dataframe is sorted appropriately.}
\CommentTok{# First by embed_V (affirmative vs. negative)}
\NormalTok{data_region_based =}\StringTok{ }\NormalTok{data_region_based[}\KeywordTok{order}\NormalTok{(data_region_based}\OperatorTok{$}\NormalTok{embed_V),]}
\CommentTok{# Then by main_V}
\NormalTok{data_region_based =}\StringTok{ }\NormalTok{data_region_based[}\KeywordTok{order}\NormalTok{(data_region_based}\OperatorTok{$}\NormalTok{main_V),]}
\CommentTok{# finally by sent_index}
\NormalTok{data_region_based =}\StringTok{ }\NormalTok{data_region_based[}\KeywordTok{order}\NormalTok{(data_region_based}\OperatorTok{$}\NormalTok{sent_index),]}
\end{Highlighting}
\end{Shaded}

\section{Embedded verb region}\label{embedded-verb-region}

\subsection{Visualization}\label{visualization}

\begin{Shaded}
\begin{Highlighting}[]
\CommentTok{# Focus on the V (verb) region.}
\NormalTok{data_V =}\StringTok{ }\KeywordTok{subset}\NormalTok{(data_region_based, region }\OperatorTok{==}\StringTok{ 'embedded_V'}\NormalTok{)}

\CommentTok{# Get difference in surprisal between shika vs. no-shika.}
\NormalTok{data_V_shika =}\StringTok{ }\KeywordTok{subset}\NormalTok{(data_V, shika }\OperatorTok{==}\StringTok{ 'shika'}\NormalTok{)}
\NormalTok{data_V_no_shika =}\StringTok{ }\KeywordTok{subset}\NormalTok{(data_V, shika }\OperatorTok{==}\StringTok{ 'no-shika'}\NormalTok{)}
\NormalTok{data_V_shika}\OperatorTok{$}\NormalTok{surprisal_diff =}\StringTok{ }\NormalTok{data_V_shika}\OperatorTok{$}\NormalTok{surprisal }\OperatorTok{-}\StringTok{ }\NormalTok{data_V_no_shika}\OperatorTok{$}\NormalTok{surprisal}

\CommentTok{# Visualize the difference in surprisal increase/dicrease between affirmative vs. negative verbs.}
\NormalTok{data_V_shika }\OperatorTok\StringTok{ }
\StringTok{    }\KeywordTok{group_by}\NormalTok{(embed_V, shika_case) }\OperatorTok
\StringTok{    }\KeywordTok{summarise}\NormalTok{(}\DataTypeTok{m=}\KeywordTok{mean}\NormalTok{(surprisal_diff),}
            \DataTypeTok{s=}\KeywordTok{std.error}\NormalTok{(surprisal_diff),}
            \DataTypeTok{upper=}\NormalTok{m }\OperatorTok{+}\StringTok{ }\FloatTok{1.96}\OperatorTok{*}\NormalTok{s,}
            \DataTypeTok{lower=}\NormalTok{m }\OperatorTok{-}\StringTok{ }\FloatTok{1.96}\OperatorTok{*}\NormalTok{s) }\OperatorTok
\StringTok{    }\KeywordTok{ungroup}\NormalTok{() }\OperatorTok
\StringTok{    }\KeywordTok{ggplot}\NormalTok{(}\KeywordTok{aes}\NormalTok{(}\DataTypeTok{x=}\NormalTok{shika_case, }\DataTypeTok{y=}\NormalTok{m, }\DataTypeTok{ymin=}\NormalTok{lower, }\DataTypeTok{ymax=}\NormalTok{upper, }\DataTypeTok{width=}\FloatTok{0.4}\NormalTok{, }\DataTypeTok{fill=}\NormalTok{embed_V)) }\OperatorTok{+}
\StringTok{        }\KeywordTok{geom_bar}\NormalTok{(}\DataTypeTok{stat =} \StringTok{'identity'}\NormalTok{, }\DataTypeTok{position =} \StringTok{"dodge"}\NormalTok{) }\OperatorTok{+}
\StringTok{        }\KeywordTok{geom_errorbar}\NormalTok{(}\DataTypeTok{position=}\KeywordTok{position_dodge}\NormalTok{(}\FloatTok{0.4}\NormalTok{), }\DataTypeTok{width=}\NormalTok{.}\DecValTok{1}\NormalTok{)}
\end{Highlighting}
\end{Shaded}

\includegraphics{analysis_embedded_shika-in-main_files/figure-latex/unnamed-chunk-2-1.pdf}

\subsection{Regressions}\label{regressions}

\subsubsection{TOP}\label{top}

\begin{Shaded}
\begin{Highlighting}[]
\NormalTok{sub_data =}\StringTok{ }\KeywordTok{subset}\NormalTok{(data_V_shika, shika_case }\OperatorTok{==}\StringTok{ 'TOP'}\NormalTok{)}

\NormalTok{m =}\StringTok{ }\KeywordTok{lmer}\NormalTok{(}
\NormalTok{        surprisal_diff}
            \OperatorTok{~}\StringTok{ }\NormalTok{embed_V}
                \OperatorTok{+}\StringTok{ }\NormalTok{(}\DecValTok{1} \OperatorTok{|}\StringTok{ }\NormalTok{sent_index)}
\NormalTok{        ,}
        \DataTypeTok{data=}\NormalTok{sub_data}
\NormalTok{        )}
\KeywordTok{summary}\NormalTok{(m)}
\end{Highlighting}
\end{Shaded}

\begin{verbatim}
## Linear mixed model fit by REML. t-tests use Satterthwaite's method [
## lmerModLmerTest]
## Formula: surprisal_diff ~ embed_V + (1 | sent_index)
##    Data: sub_data
## 
## REML criterion at convergence: 3978.9
## 
## Scaled residuals: 
##      Min       1Q   Median       3Q      Max 
## -2.59434 -0.57830  0.00579  0.58077  2.00527 
## 
## Random effects:
##  Groups     Name        Variance Std.Dev.
##  sent_index (Intercept) 0.3973   0.6303  
##  Residual               0.1357   0.3683  
## Number of obs: 2688, groups:  sent_index, 672
## 
## Fixed effects:
##               Estimate Std. Error         df t value Pr(>|t|)    
## (Intercept) -1.035e-01  2.533e-02  6.710e+02  -4.084 4.95e-05 ***
## embed_V1    -3.755e-02  7.104e-03  2.015e+03  -5.285 1.39e-07 ***
## ---
## Signif. codes:  0 '***' 0.001 '**' 0.01 '*' 0.05 '.' 0.1 ' ' 1
## 
## Correlation of Fixed Effects:
##          (Intr)
## embed_V1 0.000
\end{verbatim}

\begin{itemize}
\tightlist
\item
  Significant negative effect of embed\_V (affirmativeness = 1).

  \begin{itemize}
  \tightlist
  \item
    Negative verbs cause more
  \end{itemize}
\end{itemize}

\subsubsection{DAT}\label{dat}

\begin{Shaded}
\begin{Highlighting}[]
\NormalTok{sub_data =}\StringTok{ }\KeywordTok{subset}\NormalTok{(data_V_shika, shika_case }\OperatorTok{==}\StringTok{ 'DAT'}\NormalTok{)}

\NormalTok{m =}\StringTok{ }\KeywordTok{lmer}\NormalTok{(}
\NormalTok{        surprisal_diff}
            \OperatorTok{~}\StringTok{ }\NormalTok{embed_V}
                \OperatorTok{+}\StringTok{ }\NormalTok{(}\DecValTok{1} \OperatorTok{|}\StringTok{ }\NormalTok{sent_index)}
\NormalTok{        ,}
        \DataTypeTok{data=}\NormalTok{sub_data}
\NormalTok{        )}
\KeywordTok{summary}\NormalTok{(m)}
\end{Highlighting}
\end{Shaded}

\begin{verbatim}
## Linear mixed model fit by REML. t-tests use Satterthwaite's method [
## lmerModLmerTest]
## Formula: surprisal_diff ~ embed_V + (1 | sent_index)
##    Data: sub_data
## 
## REML criterion at convergence: 1811.6
## 
## Scaled residuals: 
##     Min      1Q  Median      3Q     Max 
## -3.5575 -0.3278  0.0332  0.4594  2.8311 
## 
## Random effects:
##  Groups     Name        Variance Std.Dev.
##  sent_index (Intercept) 0.2363   0.4861  
##  Residual               0.1067   0.3266  
## Number of obs: 1536, groups:  sent_index, 384
## 
## Fixed effects:
##               Estimate Std. Error         df t value Pr(>|t|)    
## (Intercept) -4.800e-01  2.617e-02  3.830e+02  -18.34   <2e-16 ***
## embed_V1     3.747e-01  8.334e-03  1.151e+03   44.96   <2e-16 ***
## ---
## Signif. codes:  0 '***' 0.001 '**' 0.01 '*' 0.05 '.' 0.1 ' ' 1
## 
## Correlation of Fixed Effects:
##          (Intr)
## embed_V1 0.000
\end{verbatim}

\begin{itemize}
\tightlist
\item
  Significant positive effect of embed\_V (affirmativeness = 1).
\end{itemize}

\section{Main verb region}\label{main-verb-region}

\subsection{Visualization}\label{visualization-1}

\begin{Shaded}
\begin{Highlighting}[]
\CommentTok{# Focus on the V (verb) region.}
\NormalTok{data_V =}\StringTok{ }\KeywordTok{subset}\NormalTok{(data_region_based, region }\OperatorTok{==}\StringTok{ 'main_V'}\NormalTok{)}

\CommentTok{# Get difference in surprisal between shika vs. no-shika.}
\NormalTok{data_V_shika =}\StringTok{ }\KeywordTok{subset}\NormalTok{(data_V, shika }\OperatorTok{==}\StringTok{ 'shika'}\NormalTok{)}
\NormalTok{data_V_no_shika =}\StringTok{ }\KeywordTok{subset}\NormalTok{(data_V, shika }\OperatorTok{==}\StringTok{ 'no-shika'}\NormalTok{)}
\NormalTok{data_V_shika}\OperatorTok{$}\NormalTok{surprisal_diff =}\StringTok{ }\NormalTok{data_V_shika}\OperatorTok{$}\NormalTok{surprisal }\OperatorTok{-}\StringTok{ }\NormalTok{data_V_no_shika}\OperatorTok{$}\NormalTok{surprisal}


\CommentTok{# Visualize the difference in surprisal increase/dicrease between affirmative vs. negative verbs.}
\NormalTok{data_V_shika }\OperatorTok\StringTok{ }
\StringTok{    }\KeywordTok{group_by}\NormalTok{(embed_V, main_V, shika_case) }\OperatorTok
\StringTok{    }\KeywordTok{summarise}\NormalTok{(}\DataTypeTok{m=}\KeywordTok{mean}\NormalTok{(surprisal_diff),}
            \DataTypeTok{s=}\KeywordTok{std.error}\NormalTok{(surprisal_diff),}
            \DataTypeTok{upper=}\NormalTok{m }\OperatorTok{+}\StringTok{ }\FloatTok{1.96}\OperatorTok{*}\NormalTok{s,}
            \DataTypeTok{lower=}\NormalTok{m }\OperatorTok{-}\StringTok{ }\FloatTok{1.96}\OperatorTok{*}\NormalTok{s) }\OperatorTok
\StringTok{    }\KeywordTok{ungroup}\NormalTok{() }\OperatorTok
\StringTok{    }\KeywordTok{ggplot}\NormalTok{(}\KeywordTok{aes}\NormalTok{(}\DataTypeTok{x=}\NormalTok{shika_case, }\DataTypeTok{y=}\NormalTok{m, }\DataTypeTok{ymin=}\NormalTok{lower, }\DataTypeTok{ymax=}\NormalTok{upper, }\DataTypeTok{width=}\FloatTok{0.4}\NormalTok{, }\DataTypeTok{fill=}\NormalTok{embed_V}\OperatorTok{:}\NormalTok{main_V)) }\OperatorTok{+}
\StringTok{        }\KeywordTok{geom_bar}\NormalTok{(}\DataTypeTok{stat =} \StringTok{'identity'}\NormalTok{, }\DataTypeTok{position =} \StringTok{"dodge"}\NormalTok{) }\OperatorTok{+}
\StringTok{        }\KeywordTok{geom_errorbar}\NormalTok{(}\DataTypeTok{position=}\KeywordTok{position_dodge}\NormalTok{(}\FloatTok{0.4}\NormalTok{), }\DataTypeTok{width=}\NormalTok{.}\DecValTok{1}\NormalTok{)}
\end{Highlighting}
\end{Shaded}

\includegraphics{analysis_embedded_shika-in-main_files/figure-latex/unnamed-chunk-5-1.pdf}

\begin{itemize}
\tightlist
\item
  TOP

  \begin{itemize}
  \tightlist
  \item
    Increas in surprisal in every condition.
  \end{itemize}
\item
  DAT

  \begin{itemize}
  \tightlist
  \item
    Small but expected signs of changes.
  \end{itemize}
\end{itemize}

\subsection{Regressions}\label{regressions-1}

\subsubsection{TOP}\label{top-1}

\begin{Shaded}
\begin{Highlighting}[]
\NormalTok{sub_data =}\StringTok{ }\KeywordTok{subset}\NormalTok{(data_V_shika, shika_case }\OperatorTok{==}\StringTok{ 'TOP'}\NormalTok{)}

\NormalTok{m =}\StringTok{ }\KeywordTok{lmer}\NormalTok{(}
\NormalTok{        surprisal_diff}
            \OperatorTok{~}\StringTok{ }\NormalTok{embed_V }\OperatorTok{*}\StringTok{ }\NormalTok{main_V}
                \OperatorTok{+}\StringTok{ }\NormalTok{(embed_V }\OperatorTok{+}\StringTok{ }\NormalTok{main_V }\OperatorTok{|}\StringTok{ }\NormalTok{sent_index)}
\NormalTok{        ,}
        \DataTypeTok{data=}\NormalTok{sub_data}
\NormalTok{        )}
\KeywordTok{summary}\NormalTok{(m)}
\end{Highlighting}
\end{Shaded}

\begin{verbatim}
## Linear mixed model fit by REML. t-tests use Satterthwaite's method [
## lmerModLmerTest]
## Formula: 
## surprisal_diff ~ embed_V * main_V + (embed_V + main_V | sent_index)
##    Data: sub_data
## 
## REML criterion at convergence: 1604.5
## 
## Scaled residuals: 
##     Min      1Q  Median      3Q     Max 
## -4.9448 -0.3444 -0.0138  0.3504  4.8049 
## 
## Random effects:
##  Groups     Name        Variance Std.Dev. Corr       
##  sent_index (Intercept) 0.13922  0.3731              
##             embed_V1    0.01594  0.1262    0.52      
##             main_V1     0.05716  0.2391   -0.59 -0.18
##  Residual               0.01922  0.1386              
## Number of obs: 2688, groups:  sent_index, 672
## 
## Fixed effects:
##                    Estimate Std. Error         df t value Pr(>|t|)    
## (Intercept)        0.569087   0.014640 671.000343  38.873  < 2e-16 ***
## embed_V1          -0.006159   0.005556 671.000200  -1.109    0.268    
## main_V1           -0.180980   0.009603 670.999659 -18.846  < 2e-16 ***
## embed_V1:main_V1  -0.015856   0.002674 670.999947  -5.929 4.87e-09 ***
## ---
## Signif. codes:  0 '***' 0.001 '**' 0.01 '*' 0.05 '.' 0.1 ' ' 1
## 
## Correlation of Fixed Effects:
##             (Intr) emb_V1 man_V1
## embed_V1     0.451              
## main_V1     -0.559 -0.156       
## embd_V1:_V1  0.000  0.000  0.000
\end{verbatim}

\begin{itemize}
\tightlist
\item
  No significant effect of embed\_V (affirmativeness = 1).
\item
  Significant negative effect of main\_V (affirmativeness = 1).
\item
  Significant negative interaction.
\end{itemize}

\subsubsection{DAT}\label{dat-1}

\begin{Shaded}
\begin{Highlighting}[]
\NormalTok{sub_data =}\StringTok{ }\KeywordTok{subset}\NormalTok{(data_V_shika, shika_case }\OperatorTok{==}\StringTok{ 'DAT'}\NormalTok{)}

\NormalTok{m =}\StringTok{ }\KeywordTok{lmer}\NormalTok{(}
\NormalTok{        surprisal_diff}
            \OperatorTok{~}\StringTok{ }\NormalTok{embed_V }\OperatorTok{*}\StringTok{ }\NormalTok{main_V}
                \OperatorTok{+}\StringTok{ }\NormalTok{(embed_V }\OperatorTok{+}\StringTok{ }\NormalTok{main_V }\OperatorTok{|}\StringTok{ }\NormalTok{sent_index)}
\NormalTok{        ,}
        \DataTypeTok{data=}\NormalTok{sub_data}
\NormalTok{        )}
\KeywordTok{summary}\NormalTok{(m)}
\end{Highlighting}
\end{Shaded}

\begin{verbatim}
## Linear mixed model fit by REML. t-tests use Satterthwaite's method [
## lmerModLmerTest]
## Formula: 
## surprisal_diff ~ embed_V * main_V + (embed_V + main_V | sent_index)
##    Data: sub_data
## 
## REML criterion at convergence: 600.4
## 
## Scaled residuals: 
##     Min      1Q  Median      3Q     Max 
## -4.5474 -0.3487 -0.0140  0.3562  3.0530 
## 
## Random effects:
##  Groups     Name        Variance Std.Dev. Corr       
##  sent_index (Intercept) 0.068481 0.26169             
##             embed_V1    0.008776 0.09368   0.25      
##             main_V1     0.032726 0.18090  -0.56 -0.49
##  Residual               0.026540 0.16291             
## Number of obs: 1536, groups:  sent_index, 384
## 
## Fixed effects:
##                    Estimate Std. Error         df t value Pr(>|t|)    
## (Intercept)       -0.148561   0.013986 382.999963  -10.62   <2e-16 ***
## embed_V1          -0.153967   0.006335 383.000063  -24.30   <2e-16 ***
## main_V1            0.337250   0.010124 383.000058   33.31   <2e-16 ***
## embed_V1:main_V1   0.110475   0.004157 382.999925   26.58   <2e-16 ***
## ---
## Signif. codes:  0 '***' 0.001 '**' 0.01 '*' 0.05 '.' 0.1 ' ' 1
## 
## Correlation of Fixed Effects:
##             (Intr) emb_V1 man_V1
## embed_V1     0.182              
## main_V1     -0.489 -0.337       
## embd_V1:_V1  0.000  0.000  0.000
\end{verbatim}

\begin{itemize}
\tightlist
\item
  Significant negative effect of embed\_V (affirmativeness = 1).
\item
  Significant positive effect of main\_V (affirmativeness = 1).
\item
  Significant positive interaction.
\end{itemize}


\end{document}
